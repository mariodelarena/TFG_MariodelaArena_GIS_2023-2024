\apendice{Plan de Proyecto Software}

\section{Introducción}

Este anexo recoge la planificación temporal organizada por \textit{milestones} con sus correspondientes \textit{issues} incluidos en cada uno. Además, también se desarrollará la planificación económica y la viabilidad legal del proyecto.

\section{Planificación temporal}
La planificación temporal se ha realizado mediante la metodología Scrum, por lo que cada semana de trabajo se identifica con un \textit{milestone}, los cuales contienen diferentes \textit{issues} en función de los objetivos. Tras cada \textit{milestone}, se lleva a cabo una reunión con el tutor Pedro Luis Sánchez Ortega para revisar los avances y así poder planificar las tareas para el siguiente \textit{milestone}.

\subsubsection{Objetivo 1}
Fecha: 02/12/2023 - 05/12/2023

\textit{Issues:}
\begin{itemize}
    \item Investigar sobre el 'Informe de la Comisión Bioética'.
\end{itemize}

\subsubsection{Objetivo 2}
Fecha: 05/12/2023 - 17/12/2023

\textit{Issues:}
\begin{itemize}
    \item Descripción de aplicaciones de Emotiv.
\end{itemize}

\subsubsection{Objetivo 3}
Fecha: 17/12/2023 - 06/02/2024

\textit{Issues:}
\begin{itemize}
    \item Investigación y documentación sobre las actividades pertenecientes al programa de entrenamiento de Emotiv.
\end{itemize}

\subsubsection{Objetivo 4}
Fecha: 06/02/2024 - 16/02/2024

\textit{Issues:}
\begin{itemize}
    \item Investigación sobre Emotiv Insight para comprobar las diferencias entre sus versiones, así como su alcance y uso.
\end{itemize}

\subsubsection{Objetivo 5}
Fecha: 16/02/2024 - 08/03/2024

\subsubsection{Objetivo 6}
Fecha: 08/03/2024 - 18/03/2024

\textit{Issues:}
\begin{itemize}
    \item Documentación del manual de instalación y conexión del dispositivo Emotiv MN8.
\end{itemize}

\subsubsection{Objetivo 7}
Fecha: 18/03/2024 - 25/03/2024

\textit{Issues:}
\begin{itemize}
    \item Investigación y documentación del dispositivo MN8, incluyendo sus características, las ventajas que presenta y las aplicaciones en las que se puede emplear.
    \item Realización de las sesiones de entrenamiento con MN8 mientras se documentan los resultados, problemas y soluciones experimentados.
\end{itemize}

\subsubsection{Objetivo 8}
Fecha: 25/03/2024 - 08/04/2024

\textit{Issues:}
\begin{itemize}
    \item Documentación de artículos sobre estudios realizados con Emotiv, disponibles en la página oficial de Emotiv.
\end{itemize}

\subsubsection{Objetivo 9}
Fecha: 08/04/2024 - 15/04/2024

\textit{Issues:}
\begin{itemize}
    \item Realización de la documentación de la memoria recopilada hasta el momento en la plantilla oficial proporcionada en GitHub.
    \item Realización de la documentación de los anexos recopilada hasta el momento en la plantilla oficial proporcionada en GitHub.
\end{itemize}

\subsubsection{Objetivo 11}
Fecha: 15/04/2024 - 25/04/2024

\textit{Issues:}
\begin{itemize}
    \item Realización de investigación inicial sobre Psycho Py y sus aplicaciones.
    \item Realización programa 9 días de entrenamiento proporcionado en la página web de Contour.
\end{itemize}

\subsubsection{Objetivo 12}
Fecha: 15/04/2024 - 09/07/2024

\textit{Issues:}
\begin{itemize}
    \item Realización completa de la memoria del TFG empleando la plantilla proporcionada por la UBU.
    \item Realización completa de los anexos del TFG empleando la plantilla proporcionada por la UBU.
\end{itemize}

\subsubsection{Objetivo 13}
Fecha: 25/04/2024 - 29/04/2024

\textit{Issues:}
\begin{itemize}
    \item Investigación sobre las diferencias del uso del dispositivo Emotiv Insight en PsychoPy respecto con Emotiv 'MN8'.
\end{itemize}

\subsubsection{Objetivo 14}
Fecha: 29/04/2024 - 06/05/2024

\textit{Issues:}
\begin{itemize}
    \item Investigación sobre la posible implementación de un dispositivo 'Eye Tracking' en el proyecto.
    \item Investigación sobre la validez de PyschoPy, con el objetivo de comprobar que es un programa totalmente compatible y permite obtener unos resultados válidos en sus experimentos empleando un EEG.
\end{itemize}

\subsubsection{Objetivo 15}
Fecha: 06/05/2024 - 13/05/2024

\textit{Issues:}
\begin{itemize}
    \item Búsqueda de dispositivos y versiones de eye tracking compatibles con la versión empleada de PsychoPy.
\end{itemize}

\subsubsection{Objetivo 16}
Fecha: 13/05/2024 - 20/05/2024

\textit{Issues:}
\begin{itemize}
    \item Diseño de experimento en PsychoPy similar a los ya creados en EMOTIV LABS, concretamente comienzo del experimento para la actividad 'Mindful Meditation'.
    \item Realización de pruebas utilizando un simulador de 'eye tracking' sin la necesidad de emplear el Tobii eye tracker 5.
\end{itemize}

\subsubsection{Objetivo 17}
Fecha: 20/05/2024 - 27/05/2024

\textit{Issues:}
\begin{itemize}
    \item Diseño de experimento sencillo en PsychoPy para comprender el uso de elementos básicos.
\end{itemize}

\subsubsection{Objetivo 18}
Fecha: 27/05/2024 - 03/06/2024

\textit{Issues:}
\begin{itemize}
    \item Diseño de la parte gráfica de un experimento de PsychoPy que sea lo más similar posible a la actividad de \textit{Mindful Meditation} de Emotiv Labs.
\end{itemize}

\subsubsection{Objetivo 19}
Fecha: 03/06/2024 - 10/06/2024

\textit{Issues:}
\begin{itemize}
    \item Búsqueda e investigación de algún repositorio de GitHub que contenga información relacionada con PsychoPy y Emotiv (concretamente el dispositivo MN8), bien sean características u otros experimentos realizados con ellos.
\end{itemize}

\subsubsection{Objetivo 20}
Fecha: 10/06/2024 - 17/06/2024

\textit{Issues:}
\begin{itemize}
    \item Añadir el \textit{EEG recording} al experimento de PsychoPy creado, similar al 'Mindful Meditation' de Emotiv Labs.
\end{itemize}

\subsubsection{Objetivo 21}
Fecha: 17/06/2024 - 24/06/2024

\textit{Issues:}
\begin{itemize}
    \item Compartir con el tutor el manual de instalación de PsychoPy realizado por mi para instalar esta aplicación en el ordenador del laboratorio.
    \item Realizar un prototipo de la presentación del trabajo con toda la información recopilada hasta el momento.
\end{itemize}

\subsubsection{Objetivo 22}
Fecha: 24/06/2024 - 02/07/2024

\textit{Issues:}
\begin{itemize}
    \item Investigar cómo funciona PsychoPy en cuanto a Python, es decir, si es necesario tener instalado algún entorno para realizar experimentos a parte de PsychoPy o no.
    \item Investigar sobre los casos de uso que tiene PsychoPy, tratando de documentarlo en los anexos del proyecto.
\end{itemize}

\subsubsection{Objetivo 23}
Fecha: 02/07/2024 - 05/07/2024

\textit{Issues:}
\begin{itemize}
    \item Buscar en distintos foros sobre el error obtenido en el experimento con el MN8 para tratar de solucionarlo.
\end{itemize}

\subsubsection{Objetivo 24}
Fecha: 05/07/2024 - 08/07/2024

\textit{Issues:}
\begin{itemize}
    \item Realizar el experimento que ha dado errores, pero en lugar de hacerlo con el PsychoPy Coder, se tratará de conseguir resultados adecuados con el PsychoPy Builder.
\end{itemize}

\subsection{Planificación económica}
En la planificación económica del proyecto, se analizan los costes de hardware, software y de personal de manera individual y después se calculará el coste total. Hay que tener en cuenta que los distintos precios se han obtenido en junio de 2024, ya que en el futuro pueden existir variaciones de estos.

\subsubsection{Costes de hardware}
Para este proyecto se han empleado 2 dispositivos de hardware: un ordenador portátil y el dispositivo Emotiv MN8 \cite{EmotivShop} \footnote{Página web de la tienda de Emotiv con todos los artículos de su catálogo y sus precios correspondientes \cite{EmotivShop}.}. Para calcular los costes amortizados, se ha tenido en cuenta que se posee el ordenador desde hace 3 años, respresentando estos costes en la tabla \ref{tab:costes_hardware}:

\begin{table}[H]
\centering
\begin{tabular}{|l|l|}
\hline
\rowcolor[HTML]{BFBFBF} 
\textbf{Dispositivo} & \textbf{Costes} \\ \hline
Ordenador portátil & 600€ \\ \hline
Dispositivo MN8 & 373€ \\ \hline
\textbf{Total} & 972€ \\ \hline
\end{tabular}
\caption{Costes de dispositivos hardware.}
\label{tab:costes_hardware}
\end{table}

\subsubsection{Costes de software}
Para los costes de software, hay que tener en cuenta las licencias empleadas. Para ello, en mi ordenador personal he utilizado la versión gratuita de EmotivPRO y en el laboratorio se ha empleado la licencia Estandard, observando su precio en la tabla .

\begin{table}[h]
\centering
\begin{tabular}{|l|l|}
\hline
\rowcolor[HTML]{BFBFBF} 
\textbf{Software} & \textbf{Costes} \\ \hline
EmotivPRO Standard &  139€/mes \\ \hline
\textbf{Costes 7 meses} & 975€ \\ \hline
\end{tabular}
\caption{Tabla con costes de software.}
\label{tab:costes_software}
\end{table}

\subsubsection{Costes de personal}
Para calcular los costes de personal se ha utilizado el salario de un ingeniero biomédico \cite{SueldoIngBiomed} \footnote{Página que proporciona información sobre los sueldos de diferentes profesiones \cite{SueldoIngBiomed}.}, por su similitud con el grado estudiado, el valor de IRPF \cite{IRPF2024} \footnote{Página de Bankinter que contiene información de los tramos de IRPF de todas las comunidades autónomas \cite{IRPF2024}.} y el porcentaje de la seguridad social \cite{SeguridadSocial} \footnote{Página que contiene los porcentajes de cotización a la Seguridad Social del trabajador y empresa \cite{SeguridadSocial}.}. Estos costes se pueden ver en la tabla \ref{tab:costes_personales}.

\begin{table}[H]
\centering
\begin{tabular}{|l|l|}
\hline
\rowcolor[HTML]{BFBFBF} 
\textbf{Criterio} & \textbf{Costes} \\ \hline
Salario bruto mensual & 2727€ \\ \hline
IRPF & 18,50\% \\ \hline
Seguridad Social & 6,47\% \\ \hline
Salario neto mensual & 2.046,07€ \\ \hline
\textbf{Costes 7 meses} & 14.322,5€ \\ \hline
\end{tabular}
\caption{Tabla con costes personales.}
\label{tab:costes_personales}
\end{table}

\subsubsection{Coste total}
Por lo tanto, el coste total del proyecto viene representado en la tabla \ref{tab:coste_total}.

\begin{table}[H]
\centering
\begin{tabular}{|l|l|}
\hline
\rowcolor[HTML]{BFBFBF} 
\textbf{Criterio} & \textbf{Costes} \\ \hline
Costes de hardware & 972€ \\ \hline
Costes de software & 975€ \\ \hline
Costes de personal & 14.322,5€ \\ \hline
\textbf{Coste total} & 16.269,5€ \\ \hline
\end{tabular}
\caption{Tabla con el coste total.}
\label{tab:coste_total}
\end{table}

\subsection{Viabilidad legal}

Para la viabilidad legal, hay que tener en cuenta que el dispositivo tiene que ser completamente seguro sin afectar negativamente al usuario, así como sus datos personales. Para ello, existen legislaciones que hay que cumplir para conseguir un dispositivo seguro y regulado.

\begin{itemize}
    \item Ley 24/2015 \cite{LeyPatentes}, Ley de Patentes, dónde se regula todo lo relacionado con la protección de invenciones empleando patentes, desde el registro de las patentes, invenciones patentables, el derecho a la patente y los procedimientos para pedir una patente.
    \item Real Decreto Legislativo 1/1996 \cite{LeyPropiedadIntelectual} relativo Ley de Propiedad Intelectual que regulariza la protección del derecho de autor y de derechos similares.
    \item Los productos sanitarios se rigen por la Agencia española de medicamentos y productos sanitarios (AEMPS) \cite{LegislacionProductosSanitarios}. En este proyecto nos interesan especialmente el Real Decreto 1591/2009 \cite{LeyProductosSanitarios} que regula todo lo relativo a los productos sanitarios, desde su desarrollo a su venta, y el Real Decreto 437/2002\cite{LeyLicenciasFuncionamiento} establece las pautas para la concesión de licencias de fabricación y desarrollo de productos sanitarios.
    \item Durante el desarrollo del producto se deberá cumplir con la normativa laboral española \cite{LeyesLaborales}, que incluye leyes y reglamentos como pueden ser el Estatuto de los Trabajadores, la Ley de Prevención de Riesgos Laborales o la Ley de Igualdad.
    \item Ley Orgánica 3/2018 \cite{LeyProteccionDatos} de Protección de Datos Personales y garantía de los derechos digitales. Para poder proteger cualquier información que identifique a una persona, de forma confidencial. Además, el usuario debe estar correctamente informado del tratamiento de sus datos, además el acceso al tratamiento de sus datos debe ser claro y accesible. El usuario tendrá derecho al acceso de sus datos, derecho de rectificación y supresión de sus datos, derecho a la limitación del tratamiento de sus datos, derecho a la portabilidad de sus datos y el derecho a oponerse al tratamiento de sus datos. Por todo ello el tratamiento de sus datos debe ser tras la confirmación clara del consentimiento informado del tratamiento de sus datos.
\end{itemize}

En cuanto al dispositivo Emotiv, no hay ninguna ley relacionada, pero sí que hay información sobre dispositivos EEG. Existen dos artículos que tratan la reciente legislación en el estado de Colorado y Estados Unidos, que protege la privacidad de las ondas cerebrales humanas. Esta legislación, conocida como HB-1054, es la primera de su tipo en Estados Unidos y tiene como objetivo proteger los neuroderechos de las personas.

El artículo de MVS Noticias destaca que la ley fue aprobada con una mayoría aplastante en la Cámara de Representantes de Colorado y por unanimidad en el Senado del estado. La ley amplía la definición de "datos sensibles" para incluir datos biológicos y "neuronales" generados por el cerebro, la médula espinal y la red de nervios que transmiten mensajes por todo el cuerpo. \footnote{Artículo que aprueba la primera ley del mundo en Colorado que protege la privacidad de las ondas cerebrales humanas\cite{confilegal_priv}.}

Por otro lado, el artículo de Confilegal enfatiza que esta legislación es un hito en el campo de la privacidad de datos y la neurotecnología. \footnote{Artículo sobre la cultura de privacidad ante brechas de seguridad \cite{confilega_cul}.}

 Señala que la ley apunta a la neurotecnología utilizada por los consumidores y cubre un vacío en la ley federal de salud, que protege los datos personales sensibles recopilados por dispositivos médicos en ámbitos clínicos, pero es menos estricta con los datos generados por productos neurotecnológicos de consumo. \footnote{Artículo en el que la EU aprueba ley para la protección de datos del cerebro humano\cite{mvsnoticias}.}
 
Ambos artículos destacan la importancia de esta nueva legislación en la protección de los neuroderechos y la privacidad de las ondas cerebrales humanas, ya que podría sentar un precedente para futuras leyes en otros estados y países.
