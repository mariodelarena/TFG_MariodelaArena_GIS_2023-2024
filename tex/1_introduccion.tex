\capitulo{1}{Introducción}

Este proyecto se centra en el estudio y uso del dispositivo Emotiv MN8, un dispositivo de electroencefalografía (EEG) innovador. El objetivo principal es realizar una serie de pruebas para comprender a fondo su funcionamiento y los datos que adquiere y analiza.

En la primera fase del proyecto, se realizarán pruebas con el dispositivo MN8 y las aplicaciones asociadas, Emotiv y Contour. Estas pruebas permiten familiarizarse con las capacidades del dispositivo y los métodos de recopilación y análisis de datos.

Una vez comprendido el funcionamiento del dispositivo y las aplicaciones, el siguiente paso es diseñar y realizar una serie de experimentos que impliquen el uso del EEG Emotiv MN8. Estos experimentos estarán diseñados para crear situaciones controladas, lo que permitirá recoger y analizar los resultados obtenidos.

Finalmente, el objetivo es evaluar la validez del dispositivo MN8 para su uso en el ámbito médico. Se espera que este proyecto proporcione una valiosa contribución a la comprensión de cómo se pueden utilizar los dispositivos EEG en la práctica médica y cómo los datos recogidos pueden ser utilizados para informar y mejorar las prácticas y estrategias.

Este proyecto representa una oportunidad para explorar el potencial de la tecnología EEG y su aplicación en el ámbito médico.
