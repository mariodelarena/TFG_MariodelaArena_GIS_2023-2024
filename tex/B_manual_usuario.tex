\apendice{Documentación de usuario}

\section{Requisitos software y hardware para ejecutar el proyecto.}
En esta parte del anexo, se especifican una serie de requisitos de software y hardware indispensables para el objeto del estudio. La medida del éxito del dispositivo dependerá dell número de requisitos que el prototipo logre cumplir.

\subsection{Requisitos de software}
Para utilizar la aplicación de EMOTIV, se requieren los siguientes requisitos de software:

\begin{itemize}
    \item EMOTIV Launcher. Es necesario tener instalado EMOTIV Launcher en el ordenador, el cuál permite descargar todas las aplicaciones de EMOTIV.
    \item Sistema Operativo. EMOTIV Launcher es compatible con sistemas operativos Windows y Mac.
    \item Java. Para Windows, se requiere la versión de software Java 1.6.0_17 o superior. Para Mac, se necesita Java SE 6 o superior.
 \end{itemize}   

\subsection{Requisitos de hardware}
Durante el proceso de conexión del dispositivo MN8 con el ordenador, se experimentó un problema técnico. El dispositivo MN8 se conectaba correctamente al ordenador, pero dentro de la configuración y posicionamiento del dispositivo, la calidad de la señal era del 0%. Esto implicaba que, a pesar de que el dispositivo era detectado, no había una comunicación efectiva entre el dispositivo y el ordenador.

Para resolver este problema, se realizó un cambio de sensores por los que tienen 3 filamentos, lo que aumentó la superficie de contacto. Este cambio permitió finalmente una calidad de conexión óptima del 100%.

Tras el cambio de sensores, se obtuvo una correcta solución al problema. La calidad de la conexión entre el dispositivo MN8 y el ordenador mejoró significativamente, pasando de un 0% a un 100%. Esto permitió una comunicación efectiva entre el dispositivo y el ordenador, facilitando así la realización de las tareas necesarias. Por lo tanto, se concluye que el cambio de sensores fue una solución efectiva para el problema experimentado.


\section{Instalación / Puesta en marcha}
Para utilizar el dispositivo MN8 - 2 Channel EEG Earbuds, es imprescindible la instalación de la aplicación ‘Contour’. Esta aplicación permite la vinculación con el dispositivo y la recepción en tiempo real de información sobre la actividad cerebral durante la realización de diversas actividades. A continuación, se detallan los pasos para la descarga e instalación de la aplicación:

\begin{enumerate}
    \item Acceder a la página oficial de EMOTIV y dirigirse al apartado dentro de ‘Headsets’ de MN8 – 2 Channel Wireless EEG Earbuds.
    \item Desplazarse hasta el apartado ‘Meet Contour’ y descargar la aplicación correspondiente a su sistema operativo (macOS, Windows 10, Android 7.0 o iOS 13).
    \item Una vez descargado el instalador, abrirlo y seleccionar la ruta del directorio donde desea que se instale la aplicación y se almacenen los datos de esta.
    \item Una vez descargado el instalador, abrirlo y seleccionar la ruta del directorio donde desea que se instale la aplicación y se almacenen los datos de esta.
    \item A continuación, se selecciona la fecha de nacimiento (mes y año), género, ocupación y país. Cuando estos pasos hayan sido completados, se puede seleccionar el dispositivo MN8 - 2 Channel EEG Earbuds una vez que haya sido emparejado y conectado con el ordenador.
\end{enumerate}



\section{Manuales y/o Demostraciones prácticas}
