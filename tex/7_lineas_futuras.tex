\capitulo{7}{Lineas de trabajo futuras}

Este proyecto emplea un material muy novedoso, ya que se trata de un proyecto de investigación con EMOTIV que se ha comenzado a comercializar recientemente. Por lo tanto, se podrán realizar diferentes modificaciones para mejorarlo. Algunas de las mejoras que se proponen son:

\begin{itemize}
    \item Resolver el error de conexión del dispositivo Emotiv con PsychoPy.
    \begin{itemize}
        \item A pesar de haber seguido las instrucciones tanto de PsychoPy como de Emotiv y otros tutoriales, cuando se ejecuta el experimento empleando el componente de \textit{Emotiv Reocrding}, aparece una pantalla en negra que impide la continuación del experimento. Esto es debido a que el experimento trata de conectar con el dispositivo de EEG pero no lo consigue. Este error no se explica en ninguna documentación, ni de Emotiv ni de PsychoPy, por lo que se dejará como línea futura encontrar una solución a esto.
    \end{itemize}
    \item Diversificación de Experimentos.
    \begin{itemize}
        \item Ampliar la variedad de experimentos o actividades disponibles para los usuarios, lo que evitará que los estudios se vuelvan monótonos o repetitivos. Además, se podría explorar nuevas actividades que permitan evaluar diferentes aspectos cognitivos y emocionales.
    \end{itemize}
    \item Variabilidad de Estímulos.
    \begin{itemize}
        \item Investigar para emplear una mayor variabilidad en los estímulos. Esto incluye ajustar la duración y el orden de los estímulos para evitar que los usuarios se acostumbren a patrones específicos. 
    \end{itemize}
    \item Calibración Personalizada.
    \begin{itemize}
        \item Implementar una calibración individualizada para cada usuario, lo cual adaptará los umbrales de detección de estrés y concentración según las características personales de cada usuario.
    \end{itemize}
    \item Registro de Parámetros Fisiológicos Adicionales.
    \begin{itemize}
        \item Integrar dispositivos para medir otros parámetros fisiológicos, como la frecuencia cardíaca, la temperatura corporal o el seguimiento ocular. Esto proporcionará una visión más completa de las respuestas del usuario durante los experimentos.
    \end{itemize}
    \item Base de Datos Anonimizada.
    \begin{itemize}
        \item Crear una base de datos para almacenar todos los datos recopilados por el dispositivo Emotiv MN8 manteniendo los datos de los pacientes de manera anónima para respetar la privacidad del usuario. Esta base de datos permitirá un acceso, gestión y modificación más sencillos de los datos.
    \end{itemize}
    \item Integración de R para Análisis Estadístico.
    \begin{itemize}
        \item Incorporar el lenguaje de programación R para realizar análisis estadísticos más avanzados. Se puede utilizar R para procesar los datos recopilados por el Emotiv MN8 y explorar visualizaciones y gráficos en R para presentar los resultados de manera práctica.
    \end{itemize}
    \item Realización de sesión de meditación en español.
    \begin{itemize}
        \item Se puede hablar con un profesional o especialista en locución o narración para que realice la sesión de meditación de la manera más profesional y adecuada posible.
    \end{itemize}
\end{itemize}

Por lo tanto, se puede continuar mejorando el proyecto para lograr resultados más completos y precisos, abiertos a introducir modificaciones y correcciones.
